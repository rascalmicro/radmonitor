\chapter{Introduction}

\section{Project goals}

Conducting wide-area radiation surveillance is a challenging problem for
environmental and security applications. For example, a city may wish to produce
a map of man-made gross counts over a wide region, identify
unexpected or unauthorized radioactive sources and take corrective action
\cite{Wasiolek:2007vn}. Dedicated systems have been developed, such as the
United States Department of Energy's Aerial Measuring System, which uses
aircraft to map radiological activity at nuclear sites and during emergencies
\cite{Jobst}.

It may be useful to produce regular radiation maps of an area and immediately
detect anomalies in gamma spectra, locating lost or stolen radioactive
materials, unexpected natural radioactive sources, or radiological dispersal
devices (dirty bombs). Current systems do not meet these goals: some produce
one-time static maps, rather than providing continuous monitoring, and others do
not take advantage of spatial information to more effectively detect anomalies.

We aim to construct a monitoring system which can cost-effectively map wide
areas, building maps of observed gamma spectra and rapidly detecting changes in
gamma spectra caused by new radioactive sources. To satisfy this goal, we have
constructed an integrated radiation mapping system, consisting of a mobile data
collection apparatus, spatio-temporal database of gamma spectra, and
spatially-aware anomaly mapping algorithm. We refer to this system as the
spectral comparison ratio anomaly mapping (SCRAM) system.

The SCRAM system uses mobile detectors to build spectral maps of wide-areas and
identifies temporal anomalies using spectral comparison ratios (SCRs), detecting
changes in spectral shape. By using a spatial database of observed spectra and
comparing new observations to the recorded background, a high sensitivity to
temporal anomalies is obtained.

We have tested the SCRAM system in real data collected at the Pickle Research
Campus, along with simulations of injected radioactive sources and blind tests
locating radioactive sources at large public events. The SCRAM system performs
well, locating simulated sources at long distances with small, inexpensive
detectors and detecting real medical sources in crowds at up to twenty meters.

The SCRAM system can be deployed on vehicles that naturally transit areas of
interest through their daily operations, such as patrol vehicles, municipal
buses, or unmanned aerial vehicles.  By using multiple passes of preexisting
vehicles, we expect to be able to provide high-sensitivity continuous wide-area
surveillance at minimal operational cost.

\section{Prior work}
In recent years there has been a heavy focus on radiation monitoring systems for
the detection of special nuclear material (SNM), usually at ports and border
crossings.\cite{Runkle:2005,Boardman:2012ca} These systems are usually portal
monitors: cargo, such as trucks or cargo containers, travels through detectors
as part of the customs process, and suspicious cargo is flagged for further
inspection. Improved techniques to distinguish between SNM and benign
radioactive sources are being developed, and portal monitors are routinely
deployed at American borders.

Techniques for monitoring wide areas have been explored less. General Electric's
Research Center has developed a Standoff Radiation Imaging System which uses a
large array of gamma detectors to provide gamma imaging at distances up to 100
meters; however, this requires a large and heavy apparatus with nearly 400
photomultiplier tubes and two dozen sodium iodide scintillators contained in a
commercial cargo van.\cite{Zelakiewicz:2011ig} 

More practical approaches have been tested by Pacific Northwest National Labs,
using commercial scintillators in a van which report an anomaly whenever the
spectral shape changes too rapidly.\cite{Pfund:2007,Pfund:2010hm} The
slowly-changing natural background is tracked with Kalman filters or an
exponentially-weighted moving average. This method is flexible enough to allow
for the automated rejection of nuisance sources (such as medical isotopes), but
performs no mapping and has no spatial awareness: the proposed algorithm simply
compares newly observed spectra to recently observed spectra, without
considering the motion or location of the vehicle. This makes the algorithm
oblivious to small or distant sources, which are easily confused with slow
changes in the natural background spectrum.

Current mapping techniques, however, are impractical. The most common method
uses helicopter- or aircraft-mounted gamma scintillators flown in a regular
pattern over a target area, at considerable expense.\cite{Wasiolek:2007vn}
Mapping of a large city may take days or weeks, rendering routine anomaly
detection expensive and impractical. We would like to combine a practical
mapping technique with previously developed spectral anomaly detection methods
to produce a more sensitive and useful system.

\section{Scope}

In this thesis, I describe four steps on the path to a practical wide-area
radiation surveillance system:

\begin{enumerate}
  \item Construction of a mobile spectral data collection system.
  \item Development of a spectral anomaly detection system.
  \item Development of a simulation system capable of simulating multiple
    radioactive sources and varying natural background radiation.
  \item Evaluation and testing of the system with data collection at the Pickle
    Research Campus and public events.
\end{enumerate}

Some work remains to produce a truly practical field-ready system, such as the
development of battery-powered lightweight mobile data collection units,
wireless data transmission and collection, and real-time analysis and
mapping. Applied Research Laboratories intends to pursue these goals in future
work. 

In Chapter~\ref{kriging}, I discuss a possible alternative algorithm to the one
discussed and developed in this thesis. The alternative method, making use of
geostatistical models such as kriging, has promise for more practical and useful
anomaly mapping systems, but has not yet been fully developed. I have developed
a pilot system and performed preliminary simulation studies to characterize its
behavior.
