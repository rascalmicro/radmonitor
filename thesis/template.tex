\documentclass[letterpaper,11pt,oneside]{memoir} % remove 'oneside' for a
                                % version suitable for printing and binding
\usepackage[utf8]{inputenc}
\usepackage[T1]{fontenc}
\usepackage{graphicx}
% Improved fonts
% See also http://www.tug.dk/FontCatalogue/
\usepackage{mathpazo} % Palatino fonts for mathematics
\usepackage{tgpagella} % improved Palatino font for text

\usepackage{microtype} % better justification
\usepackage{amsmath}
\usepackage{cmap} % make PDFs searchable

% hyperref turns internal references to hyperlinks, allows \url{}
% also allows setting PDF meta information
\usepackage[pdftex,
  pdfauthor={Your Name Here},
  pdftitle={Your Title Here}]{hyperref}

%% Memoir options
%% The Memoir manual is available online:
%% http://www.tex.ac.uk/tex-archive/macros/latex/contrib/memoir/memman.pdf
\chapterstyle{ell}
% Put section numbers in left margin
\setsecindent{-33pt}
% Include subsections in TOC
\setcounter{tocdepth}{3}
\setsecnumdepth{subsection}
% Set caption fonts
\captionnamefont{\small\itshape} % make "Figure n.n:" small and italic
\captiontitlefont{\small} % make the caption text small
\captionstyle[\centering]{\raggedright}

% Signature line
\newcommand{\sigline}{\noindent
\makebox[2in]{\hrulefill}\\}

\begin{document}
\pagenumbering{roman}

% Title page
\begin{titlingpage}
\begin{center}
\begin{Spacing}{2}
{\huge
  An Example Physics Thesis
}
\end{Spacing}
\end{center}

\begin{center}
  Presented by Your Name\\
  The University of Texas at Austin
\end{center}

\begin{center}
In partial fulfillment of the requirements for graduation with the\\
Dean’s Scholars Honors Degree in Physics
\end{center}

\vskip 1in

\begin{minipage}[t]{0.4\textwidth}
\sigline
Prof.~John Smith\\
Supervising Professor
\end{minipage}\hskip 1in
\begin{minipage}[t]{0.4\textwidth}
\sigline
Date
\end{minipage}

\vskip 0.75in

\begin{minipage}[t]{0.4\textwidth}
\sigline
Prof.~Greg Sitz\\
Honors Adviser in Physics
\end{minipage}\hskip 1in
\begin{minipage}[t]{0.4\textwidth}
\sigline
Date
\end{minipage}
\end{titlingpage}

\cleardoublepage

\chapter*{Acknowledgments}

Lorem ipsum dolor sit amet, consectetur adipiscing elit. Integer ultricies,
sapien eu sagittis aliquam, lectus dui faucibus leo, id euismod tellus nisl et
augue. Sed tempor tristique tortor, sed auctor nisi consequat quis. Morbi
pulvinar purus in tortor venenatis eu placerat dui scelerisque. Quisque rutrum
felis quis diam tincidunt et feugiat magna eleifend. Fusce vehicula tincidunt
semper. Suspendisse dapibus porttitor purus ac suscipit. Ut gravida justo nec
metus ornare sed vehicula lorem porta. Phasellus pellentesque imperdiet dolor,
ut rhoncus odio cursus nec. Proin non orci justo, id volutpat turpis.

% Move to the next right-hand page; if printed, you want chapters to start on
% the right side.
\cleardoublepage

\chapter*{Abstract}
Lorem ipsum dolor sit amet, consectetur adipiscing elit. Integer ultricies,
sapien eu sagittis aliquam, lectus dui faucibus leo, id euismod tellus nisl et
augue. Sed tempor tristique tortor, sed auctor nisi consequat quis. Morbi
pulvinar purus in tortor venenatis eu placerat dui scelerisque. Quisque rutrum
felis quis diam tincidunt et feugiat magna eleifend. Fusce vehicula tincidunt
semper. Suspendisse dapibus porttitor purus ac suscipit. Ut gravida justo nec
metus ornare sed vehicula lorem porta. Phasellus pellentesque imperdiet dolor,
ut rhoncus odio cursus nec. Proin non orci justo, id volutpat turpis.

\cleardoublepage

\tableofcontents

\cleardoublepage

\pagenumbering{arabic}

\chapter{Introduction}

Lorem ipsum dolor sit amet, consectetur adipiscing elit. Integer ultricies,
sapien eu sagittis aliquam, lectus dui faucibus leo, id euismod tellus nisl et
augue. Sed tempor tristique tortor, sed auctor nisi consequat quis. Morbi
pulvinar purus in tortor venenatis eu placerat dui scelerisque. Quisque rutrum
felis quis diam tincidunt et feugiat magna eleifend. Fusce vehicula tincidunt
semper. Suspendisse dapibus porttitor purus ac suscipit. Ut gravida justo nec
metus ornare sed vehicula lorem porta. Phasellus pellentesque imperdiet dolor,
ut rhoncus odio cursus nec. Proin non orci justo, id volutpat turpis.

% For your content, place each chapter in a separate TeX file for convenience,
% and then simply use
% \input{chaptername.tex}
% to have the text included here.
% You don't need a document preamble or anything in each chapter file -- just
% start with
% \chapter{Title here}
% and run with it. Then just run LaTeX on this file to get the entire thesis
% compiled in one go.

\end{document}