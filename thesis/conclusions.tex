\chapter{Conclusions}

In this work, we report on the development of an integrated system for wide-area
radiation surveillance.  A mobile detector is used to collect multiple-pass
data, which is stored into a spatial-temporal database.  We developed a novel
approach for anomaly detection of the spectral content by comparing observations
to previous determinations of the background.  Because the spatial variance is
much larger than the temporal variance, this multi-pass methodology has the
potential to deliver increased sensitivity to faint or distant sources.

In total, the system provides an increase in sensitivity assessment to radiation
changes, while operating in a more efficient and cost-effective manner than
dedicated mapping systems.  These developments are the first steps necessary to
implement the larger vision of a providing continuous wide-area surveillance
through the use of mobile sensors.

Some future improvements will be necessary to bring the system into practical
use. Real-time energy calibration, using a known check source carried with the
detector, may be necessary for more accurate spectral anomaly detection. Such
systems have already been developed for other gamma ray detection applications,
as scintillator crystals tend to ``drift'' and lose
calibration.\cite{Runkle:2009ev}

Data cleaning will also be necessary to use GPS data. When walking under
overhangs or in urban canyons, the estimated GPS position may drift, smearing
data over a wide area. Techniques to reduce this problem will be necessary; for
example, if the detector travels a regular route, deviations from this route
can be ``snapped'' back to the usual route. In our stadium data, it may be
possible to simply check for data points where the estimated velocity is larger
than walking speed, or where the detector position drifts across the football
field.

Future improvements center on improvements to the spatial kriging method for
anomaly detection, which shows great promise in detecting anomalies at longer
distances. Simulation studies and experimental tests will be needed to test the
viability of functional cokriging and other techniques to improve results. Other
work may focus on methods to determine optimal energy bin sizes and adaptation
for use in small, low-power mobile detectors.
